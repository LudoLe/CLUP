\section{Introduction}
\label{sect:introduction}

\subsection{Purpose}
\label{subsect:purpose}

This document represents the Requirement Analysis and Specification Document (RASD). 

Goals of this document are to completely describe the CLup system in terms of functional and non-functional requirements, to analyze the needs of the future users of our system in order to properly model it, to show the constraints and the limitations of our software and finally, to indicate the typical real world cases in which our application will actually be made use of.

This document is addressed to the \textit{ones who take part in the process of development} of the first release of CLup and its aim is to describe the \textit{interactions between the real world and the application.}

\subsection{Scope: description of the given problem}
\label{subsect:scope}

In these trying times of global pandemic, such a common matter as going grocery shopping has become a relevant threat to public health. 
Nethertheless, grocery shopping still remains an essential need which has to be carried out: being so, avoiding crowding up either inside and outside of grocery shops to avoid any source of hazards uprises as a new main issue to focus on.

CLup is a software application that will be implemented in order to face this issue. The application proposes itself to help either the grocery shop owners to adequate to the new governmental rules and grocery shop customers to protect their own health.

In fact, Clup allows customers to \textit{book online their shopping sessions} when and for how long they desire, letting them even specify the categories of items they are willing to buy: in this way the system will be able to grant the rules of social distancing more accurately and let the clients have an even safer experience through their shopping session.

Clup also allows those clients who are not much of a planner to \textit{join virtual queues}, which are meant to substitute the physical ones, for their last minute shopping sessions. The system will permit them to monitor the queue and will even alert them when it’s time to leave in order to reach the shop in time for their turn.

Finally, CLup will be of great help to grocery shop owners in \textit{regulating the incoming influx of people} in their shop. As a matter of fact, CLup implements a QR-code based system of monitoring accesses to the shops that will decide to adherit to our service. 

Also, CLup will be a convinient way to keep up with the evolving rules and law, giving managers the possibility to immediately customize every aspect of a shop.

\subsection{Definitions, acronyms and abbreviations}
\label{subsect:definitionsacronymsabbreviations}

In order to to avoid any misunderstanding here we present three tables to explain all the \textit{definitions}, \textit{acronyms} and \textit{abbreviations} used throughout the document.

\begin{table}[h!]
    \centering
    \begin{tabular}{@{}P{0.25\textwidth}P{0.6\textwidth}@{}}
        \multicolumn{2}{c}{\textbf{Definitions}}\\
        \toprule
        \textbf{Customer} & a person who is going to go grocery shopping and who is not registered in the CLup system\\
        \textbf{User} & a client registered in the CLup system\\
        \textbf{Shop owner} & a grocery shop owner or administrator who hasn't already adhered to CLup\\
        \textbf{Manager} & a grocery shop owner or administrator who adhered to CLup\\
        \textbf{Totem} & the piece of hardware that allows customers to be able to enqueue without having the app downloaded and which is placed outside of the shop it's correlated to\\
    \end{tabular}
\caption{Definitions}
\label{table:definitions}
\end{table}

\begin{table}[h!]
    \centering
    \begin{tabular}{@{}P{0.25\textwidth}P{0.6\textwidth}@{}}        
        \multicolumn{2}{c}{\textbf{Abbreviations}}\\
        \toprule
        \textbf{Gn} & $n^{th}$-goal\\
        \textbf{Dn} & $n^{th}$-domain assumption\\
        \textbf{Rn} & $n^{th}$-functional requirement\\
    \end{tabular}
\caption{Abbreviations}
\label{table:abbreviations}
\end{table}

\begin{table}[h!]
    \centering
    \begin{tabular}{@{}P{0.25\textwidth}P{0.6\textwidth}@{}}
        \multicolumn{2}{c}{\textbf{Acronyms}}\\
        \toprule
        \textbf{RASD} & Requirement Analysis and Specification Document\\
        \textbf{API} & Application Programming Interface\\
        \textbf{GPS} & Global Positioning System\\
    \end{tabular}
\caption{Acronyms}
\label{table:acronyms}
\end{table}

%bsection{Overview}
%\label{subsect:overview}

%Here we present an overview of what the next sections of the document will adress.

%TODO:The \textbf{overall description} [\ref{sect:overalldescription}] section starts presenting some common scenarios [\ref{subsect:scenarios}] that give the reader a first idea of how the system works. Then the document analyzes all the goals [\ref{subsect:goals}], domain assumption [\ref{subsect:domainassumptions}] and costraints [\ref{subsect:contraints}] of the application, followed by a list of all the actors [\ref{subsect:actors}] and an in-depth-analysis of all the use cases [\ref{subsect:usecases}], highlighting the interaction between the application and the world. This following subsections contain diagrams [\ref{subsect:systemstructure}], that encapsulate an high-level view of the structure [\ref{subsubsect:applicationstructure}] and functions [\ref{subsubsect:functionsstructure}] of the system.