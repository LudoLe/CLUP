\section{Introduction}
\label{sect:introduction}

\subsection{Purpose}
\label{subsect:purpose}

This section should include the goals.

This document represents the Requirement Analysis and Specification Document (RASD). Goals of this document are to completely describe the system in terms of functional and non-functional requirements, analyze the real needs of the customer in order to model the system, show the constraints and the limit of the software and indicate the typical use cases that will occur after the release. This document is addressed to the developers who have to implement the requirements and could be used as a contractual basis

\subsection{Scope}
\label{subsect:scope}

Scope: here we include an analysis of the world and of the shared phenomena. Identifies the product and application domain.

\subsubsection{Description of the given problem}
\label{subsect:descriptionofthegivenproblem}

In these trying times of global pandemic, such a common matter as going grocery shopping has become a relevant threat to public health. 
Nethertheless, grocery shopping still remains an essential need which has to be carried out: being so, avoiding crowding up either inside and outside of grocery shops to avoid any source of hazards becomes the new main issue to focus on.\ 
CLup is the software application we will implement in order to face this issue.
The application proposes itself to help either the grocery shop owners to adequate to the new governmental rules and grocery shop customers to protect their own health.\ 
In fact, Clup allows customers to book online their shopping session when and for how long they desire, letting them even specify the categories of items they are willing to buy: in this way the system will be able to grant the rules of social distancing more accurately and let the clients have an even safer experience through their shopping session.
Clup also allows those clients who are not much of planners to join a virtual queue, which is meant to substitute the physical one, for their last minute shopping sessions. The system will permit them to monitor the queue and will even alert them when it’s time to leave in order to reach out the shop in time for their turn. 
Indeed, Clup, using the customers’ GPS position, is able to tell the clients -only the ones who are willing to use this service, of course!- how long it will take them to get to the previously selected shop.
Finally, CLup will give much of a help to grocery shop owners in regulating the incoming influx of people in their shop. As a matter of fact, CLup implements a QR based system of monitoring accesses to the shops that will decide to adopt our application. This last function will be possible thanks to some hardware facilities.

\subsubsection{Goals}
\label{subsect:goals}

Shop Manager:
[G1] Allow a manager to sign on the system
[G2] Allow a manager to sign in the system
[G3] Allow a manager to register their store/stores on the system
    [G.3.1]Allow a manager to register basic info about the shop
	[G.3.2]Allow a manager to divide their store in areas 
	[G.3.3]Allow a manager to register the items in the areas 
[G4]Allow a manager to update the shop info
[G5]Allow a manager to check the general status of their shop

Customers:
[G6]Allow a customer to join the virtual queue from the spot*
[G7]Allow a customer to join the virtual queue from the app
[G8]Allow a user to sign on the system
[G9]Allow a registered-user to sign in the system
[G10]Allow a registered-user to book a shopping session at a grocery store
    [G9.2]Allow a registered-user to select the time and duration(between 		predefined options) for they session 	
    [G9.1]Allow a registered-user to select categories of items they are 	willing to 
    buy
[G11]Allow a registered-user to check retrieve informations about its previously booked visits
[G12]Allow a user to retrieve informations about the congestion of shops
[G12]Allow a customer to use a QR code to get access to the store 


* with the facility of a hardware support, it will be better specificed later on this document

\begin{tabular}{|l|l|}
    \hline
    \textbf{Identifier} & \textbf{goal}\\
    \hline
    \textbf{G.1} & Allow a manager to monitor the influx of clients in their store\\
    \textbf{G.1.1} & allow a manager to register his store on the system\\
    \textbf{G.1.2} & Allow a customer to access a store with a QR code\\
    \hline
    \textbf{G.2} & Allow customers to line up in a virtual queue in order to access a grocery shop\\
    \textbf{G.2.1} & allow a manager to get a numbered ticket to a client that represents their place in the queue\\
    \textbf{G.2.2} & allow a customer to get a numbered ticket that represents their place in the queue\\
    \hline
    \textbf{G.3} & Allow a customer to book a visit\\
    \textbf{G.3.1} & Allow users to indicate the categories of items that they intend to buy\\
    \textbf{G.3.2} & Allow users to indicate the time they intend to go grocery shopping\\
    \textbf{G.3.3} & Allow users to indicate the estimated permanence time \\
    \hline
\end{tabular}

\subsection{Phenomena}
\label{subsect:phenomena}

The machine: the portion of system to be developed.
The world (a.k.a. the environment): the portion of the real-world affected by the machine.
Requirements engineering is concerned with phenomena occurring in the world, as opposed to phenomena occurring inside the machine.

Goals are prescriptive assertions formulated in terms of \textbf{world} phenomena (not necessarily shared).
Domain properties/assumptions are descriptive assertions assumed to hold in the \textbf{world}.
Requirements are prescriptive assertions formulated in terms of \textbf{shared} phenomena.
Requirements are a bridge from the \textbf{machine} to the real \textbf{world}.

Find all the possible phenomena, for each one specify if its shared or not and who controls it (world or machine). 

All phenomena observed:

\begin{tabular}{|l|c|c|}
    \hline
    \textbf{Phenomenon} & \textbf{shared} & \textbf{controlled by}\\
    \hline
    User downloads the app & ? & ? \\
    Shop owner signs up to the app & ? & ? \\
    Customer signs up to the app & ? & ? \\
    Shop owner signs in & ? & ? \\
    Customer signs in & ? & ? \\
    Shop owner install a QR codes scanner & ? & ? \\
    \hline
    User wants to go to grocery store & ? & ? \\
    User searches a store in the CLup app & ? & ? \\
    User retrieve information about the queue of a shop & ? & ? \\
    User "lines up" for a store & ? & ? \\
    User searches information about his turn in the queue & ? & ? \\
    User receives notifications/updates about his position in the queue & ? & ? \\
    User receives a notification about special changes on the queue & ? & ? \\ %because of possible modifications from the manager, anche se non so se abbia senso sta cosa
    User goes to the grocery store & ? & ? \\
    USer enters the grocery store & ? & ? \\
    User shows his QR code to the store's scanner before going inside & ? & ? \\
    User does the shopping & ? & ? \\
    User shows his QR code to the store's scanner before going out & ? & ? \\
    User exits the grocery store & ? & ? \\
    
    %TODO:
    %Customers without CLup app reaches the store & ? & ? \\
    %Manager enqueue a customer & ? & ? \\
    %Manager checks esitmated queue time for customers he has put in queue & ? & ? \\
    %Manager communicates to CLup a customer has entered the store & ? & ? \\
    %Manager communicates to CLup an user ha exit the store & ? & ? \\
    
    User leaves the queue & ? & ? \\
    User doesn't show up & ? & ? \\
    User shows up early & ? & ? \\
    \hline
    User cheks available visits for a store & ? & ? \\
    User specifies the categories he is interested in & ? & ? \\
    User books a visit to the grocery store & ? & ? \\
    User checks his booked visits & ? & ? \\
    User receives notification/reminders about an imminent booked visit & ? & ? \\
    User receives notification about special changes on the visit & ? & ? \\
    Booked user is added to the queue & ? & ? \\ %idea per come implementarlo: una persona fa una prenotazione, il programma automaticamente mette in coda il cliente in modo tale che il suo turno corrispona all'orario di prenotazione
    \hline
    Manager registers a shop & ? & ? \\
    Manager setsup/updates the general features of his shops & ? & ? \\
    Manager checks status/info of the queue & ? & ? \\
    Manager checks status/info of the booked visits & ? & ? \\
    Manager edits the queue & ? & ? \\
    Manager edits the booked visits & ? & ? \\
    \hline
    The system crashes & ? & ? \\
    \hline
\end{tabular}

\textbf{DOMANDA:} cosa cambia essere registrati o meno? E' necessario registrarsi?

\subsubsection{World phenomena}
\label{subsubsect:worldphenomena}

World phenomena are phenomena that the machine can not observe.

\subsubsection{Shared Phenomena}
\label{subsubsect:sharedphenomena}

Some world phenomena are shared with the machine.
Shared phenomena can be controlled by the world and observed by the machine, or controlled by the machine andd observed by t5he world.

\subsection{Definitions, acronyms and abbreviations}
\label{subsect:definitionsacronymsabbreviations}

\subsubsection{Definitions}
\label{subsect:definitions}
Customer: a person who is going to go grocery shopping.
User: a person who downloaded Clup but who isn't registered.
Registered-user: a person who downloaded Clup and who is registered.
Shop manager/ Shop owner: a grocery shop owner or manager who decided to adopt Clup system.

\subsubsection{Acronyms}
\label{subsect:acronyms}
RASD: Requirement Analysis and Specification Document.
API: Application Programming Interface.
GPS: Global Positioning System.

\subsubsection{Abbreviations}
\label{subsect:abbreviations}
• [Gn]: n-goal.
• [Dn]: n-domain assumption.
• [Rn]: n-functional requirement.


\subsection{Revision history}
\label{subsect:revisionhistory}

\subsection{References}
\label{subsect:references}
• Specification Document: “Assignments AA 2020-2021.pdf”.
• IEEE Std 830-1998 - IEEE Recommended Practice for Software Requirements Specifications.
• GPS Performances: “http://www.gps.gov/systems/gps/performance/accuracy/".
• Alloy Dynamic Model example: “http://homepage.cs.uiowa.edu/~tinelli/classes/181/
Spring10/Notes/09-dynamic-models.pdf"
• IEEE Std 830-1993 - IEEE Guide to Software Requirements Specifications.
• IEEE Std 830-1998 - IEEE Recommended Practice for Software Requirements Specifications.
• RASD Sample A.Y. 2016-2017
• RASD Sample A.Y. 2015-2016


\subsection{Overview}
\label{subsect:overview}

Describes contents and structure of the remainder of the RASD