\section{Introduction}
\label{sect:introduction}

\subsection{Purpose}
\label{subsect:purpose}

This document represents the Requirement Analysis and Specification Document (RASD). 

Goals of this document are to completely describe the CLup system in terms of functional and non-functional requirements, analyze the real needs of the customer in order to model the system, show the constraints and the limit of the software and indicate the typical use cases that will occur after the release. 

This document is addressed to who takes part in development of the first release of CLup, and his aim is to clarify and expose all the interaction between the real world and the application.

\subsection{Scope}
\label{subsect:scope}

\subsubsection{Description of the given problem}
\label{subsect:descriptionofthegivenproblem}

In these trying times of global pandemic, such a common matter as going grocery shopping has become a relevant threat to public health. 
Nethertheless, grocery shopping still remains an essential need which has to be carried out: being so, avoiding crowding up either inside and outside of grocery shops to avoid any source of hazards becomes the new main issue to focus on.

CLup is the software application we will implement in order to face this issue. The application proposes itself to help either the grocery shop owners to adequate to the new governmental rules and grocery shop customers to protect their own health.

In fact, Clup allows customers to \textit{book online their shopping sessions} when and for how long they desire, letting them even specify the categories of items they are willing to buy: in this way the system will be able to grant the rules of social distancing more accurately and let the clients have an even safer experience through their shopping session.

Clup also allows those clients who are not much of planners to \textit{join virtual queues}, which are meant to substitute the physical ones, for their last minute shopping sessions. The system will permit them to monitor the queue and will even alert them when it’s time to leave in order to reach the shop in time for their turn.

Finally, CLup will give much of a help to grocery shop owners in \textit{regulating the incoming influx of people} in their shop. As a matter of fact, CLup implements a QR based system of monitoring accesses to the shops that will decide to adherit to our service. 

Also, CLup will be a convinient way to keep up with the evolving rules and law, giving managers the possibility to immediately customize every aspect of a shop.

\subsubsection{Main funcitons}
\label{subsect:mainfunctions}

As we mentioned in the product scope, the goals of Clup are fundamentally three: 
\begin{enumerate}[topsep=0pt]
    \item help the shop owners to \textit{regulate the influx of incoming people} in their shops;
    \item allow customers to \textit{join a virtual queue};
    \item allow customers to \textit{book shopping sessions}.
\end{enumerate}

However, it is important to recall herein that these purposes are primarily imposed by the actual situation of global pandemic, and that the true underlying goal, which defines the objective of our application, is, in first instance, to \textit{keep grocery customers as safe as possible} and, in second instance, to \textit{help as much as possible shop owners} to be able adapt to the new in force regulations.

\subsection{Definitions, acronyms and abbreviations}
\label{subsect:definitionsacronymsabbreviations}

In order to to avoid any misunderstanding here we present three table to explain all the \textit{definitions}, \textit{acronyms} and \textit{abbreviations} used throught the document.

\begin{table}[h!]
    \centering
    \begin{tabular}{@{}P{0.15\textwidth}P{0.5\textwidth}@{}}
        \multicolumn{2}{c}{\textbf{Definitions}}\\
        \toprule
        \textbf{Customer} & a person who is going to go grocery shopping and is not registered in the CLup system\\
        \textbf{User} & a customer who downloaded Clup and is registered\\
        \textbf{Shop owner} & a grocery shop owner or administrator who hasn't already adherit to CLup\\
        \textbf{Manager} & a grocery shop owner or administrator who has adherit to CLup\\
        \textbf{Totem} & TODO:\\
    \end{tabular}
\caption{Definitions}
\label{table:definitions}
\end{table}

\begin{table}[h!]
    \centering
    \begin{tabular}{@{}P{0.15\textwidth}P{0.5\textwidth}@{}}        
        \multicolumn{2}{c}{\textbf{Abbreviations}}\\
        \toprule
        \textbf{Gn} & $n^{th}$-goal\\
        \textbf{Dn} & $n^{th}$-domain assumtion\\
        \textbf{Rn} & $n^{th}$-functional requirement\\
    \end{tabular}
\caption{Abbreviations}
\label{table:abbreviations}
\end{table}

\begin{table}[h!]
    \centering
    \begin{tabular}{@{}P{0.15\textwidth}P{0.5\textwidth}@{}}
        \multicolumn{2}{c}{\textbf{Acronyms}}\\
        \toprule
        \textbf{RASD} & Requirement Analysis and Specification Document\\
        \textbf{API} & Application Programming Interface\\
        \textbf{GPS} & Global Positioning System\\
    \end{tabular}
\caption{Acronyms}
\label{table:acronyms}
\end{table}

\subsection{Overview}
\label{subsect:overview}

Here we present an overview of what the next sections of the document will adress.

%TODO:The \textbf{overall description} [\ref{sect:overalldescription}] section starts presenting some common scenarios [\ref{subsect:scenarios}] that give the reader a first idea of how the system works. Then the document analyzes all the goals [\ref{subsect:goals}], domain assumption [\ref{subsect:domainassumptions}] and costraints [\ref{subsect:contraints}] of the application, followed by a list of all the actors [\ref{subsect:actors}] and an in-depth-analysis of all the use cases [\ref{subsect:usecases}], highlighting the interaction between the application and the world. This following subsections contain diagrams [\ref{subsect:systemstructure}], that encapsulate an high-level view of the structure [\ref{subsubsect:applicationstructure}] and functions [\ref{subsubsect:functionsstructure}] of the system.

%TODO: