\section{Introduction}
\label{sect:introduction}

\subsection{Purpose}
\label{subsect:purpose}

This section should include the goals.

This document represents the Requirement Analysis and Specification Document (RASD). Goals of this document are to completely describe the system in terms of functional and non-functional requirements, analyze the real needs of the customer in order to model the system, show the constraints and the limit of the software and indicate the typical use cases that will occur after the release. This document is addressed to the developers who have to implement the requirements and could be used as a contractual basis

\subsection{Scope}
\label{subsect:scope}

Scope: here we include an analysis of the world and of the shared phenomena. Identifies the product and application domain.

\subsubsection{Description of the given problem}
\label{subsect:descriptionofthegivenproblem}

In these trying times of global pandemic, such as a common matter as going grocery shopping has become a relevant threat for the general public health. In spite of this situation, grocery shopping still remains an essential need which has to be carried out: being so, avoiding crowding up either inside and outnside of grocery shops in order to avoid any source of hazards becomes the new main issue to focus on.\ CLup is a new software application that proposes itself to help either the grocery shop owners to adequate to the new governamental rules and grocery shop customers to protect their own health.\ Clup provides three main functionalities: a virtual queue, which has the conceptual purpose of substituing the physical lines outside of shops; a QR code identification system, which allows grocery shops managers to constantly monitor the influx of incoming people in the shop and, finally, the functionality of booking your grocery shop session on an online facility, this offers a further way to regulate people flow to shops managers and the possibility to better schedule their time to cutomers.

\subsubsection{Goals}
\label{subsect:goals}

Allow a manager to monitor the influx of clients in their store 
allow a manager to register his store on the system
Allow a customer to access a store with a QR code

Allow customers to line up in a virtual queue in order to access a grocery shop
allow a manager to get a numbered ticket to a client that represents their place in the queue
allow a customer to get a numbered ticket that represents their place in the queue

Allow a customer to book a visit
Allow users to indicate the categories of items that they intend to buy;
Allow users to indicate the time they intend to go grocery shopping
Allow users to indicate the estimated permanence time 

\subsection{Phenomena}
\label{subsect:phenomena}

The machine: the portion of system to be developed.
The world (a.k.a. the environment): the portion of the real-world affected by the machine.
Requirements engineering is concerned with phenomena occurring in the world, as opposed to phenomena occurring inside the machine.

Goals are prescriptive assertions formulated in terms of \textbf{world} phenomena (not necessarily shared).

Domain properties/assumptions are descriptive assertions assumed to hold in the \textbf{world}.

Requirements are prescriptive assertions formulated in terms of \textbf{shared} phenomena.

Requirements are a bridge from the \textbf{machine} to the real \textbf{world}.

Find all the possible phenomena, than, for each one specify if its shared or not and who controls it (world or machine)

\subsubsection{World phenomena}
\label{subsubsect:worldphenomena}

World phenomena are phenomena that the machine can not observe.

\subsubsection{Shared Phenomena}
\label{subsubsect:sharedphenomena}

Some world phenomena are shared with the machine.
Shared phenomena can be controlled by the world and observed by the machine, or controlled by the machine andd observed by the world.

\subsection{Definitions, acronyms and abbreviations}
\label{subsect:definitionsacronymsabbreviations}

\subsubsection{Definitions}
\label{subsect:definitions}

\subsubsection{Acronyms}
\label{subsect:acronyms}

\subsubsection{Abbreviations}
\label{subsect:abbreviations}

\subsection{Revision history}
\label{subsect:revisionhistory}

\subsection{References}
\label{subsect:references}

\subsection{Overview}
\label{subsect:overview}

Describes contents and structure of the remainder of the RASD