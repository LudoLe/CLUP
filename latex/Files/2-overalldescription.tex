\section{Overall description}
\label{sect:overalldescription}

%TODO: eliminare
%\begin{sidewaysfigure}
%\centering
%\includegraphics[width=\textwidth]{Images/11.png}
%\caption{\label{fig:metamodel}DICE DPIM metamodel.}
%\end{sidewaysfigure}

%\begin{figure}
%\centering
%\includegraphics[width=\textwidth]{Images/11.png}
%\caption{\label{fig:metamodel2}DICE DPIM metamodel in portrait form.}
%\end{figure}

\subsection{Details about the specification}
\label{subsect:furtherdetailsaboutspecification}

The full specification can be found at the link: \url{https://github.com/LudoLe/LermaMainettiGambera/blob/master/other%20documents/R%26DD%20Assignment%20A.Y.%202020-2021%20(2).pdf}.
\label{specification}

\subsubsection{Stakeholders}
\label{subsubsect:stakeholders}

Here are listed all the potential stakeholders that will interact with CLup system:
\begin{itemize}[topsep=0pt]
    \item \textbf{Customers}: a customer is a person who is not registered in our system, and who will interact with it via the totem at the entrance of the shops.
    \item \textbf{Users}: a user is a person who is registered in our system, and who will interact with it via the mobile application or the webite.
    \item \textbf{Shop owners}: a shop owner is a person who owns a shops and is not registered in our system.
    \item \textbf{Managers}: a manager is a shop owner who has decided to adherit to our system and is registered.
\end{itemize}

\subsubsection{Main funcitons}
\label{subsect:mainfunctions}

As we mentioned in the product scope, the goals of Clup are fundamentally three: 
\begin{enumerate}[topsep=0pt]
    \item help the shop owners to \textit{regulate the influx of incoming people} in their shops;
    \item allow customers to \textit{join a virtual queue};
    \item allow customers to \textit{book shopping sessions}.
\end{enumerate}

However, it is important to recall herein that these purposes are primarily imposed by the actual situation of global pandemic, and that the true underlying goal, which defines the objective of our application, is, in first instance, to \textit{keep grocery customers as safe as possible} and, in second instance, to \textit{help as much as possible shop owners} to be able adapt to the new in force regulations.

\subsubsection{Scenarios}
\label{subsubsect:scenarios}

Here we want to present some usefull scenarios in order to describe what people will do and experience as they try to make use of CLup services. Our intent is not to specify every possible situation, but to give the reader a general understanding of the main features of the system with a concrete description of ideal cases.
 
\begin{description}
    \item[User lines up] 
    %In a period of global pandemic and quarantine, XXX needs to do the grocery shopping and, remembering the risks he used to take when there was no organization, he smiles opening the CLup app from his smartphone. He immediately searches the store he wants to go to, and in a few clicks he has all the relevant information about the status of the virtual queue and an estimation of how much time to wait. He decides to enqueue himself and by informing the app about his stay at the shop and how he'll reach it, he's ready to confirm and receive his personal QR code. Now it's time for XXX to wait in the comfort of his house, untill, when the time comes, he receives a push(?) notification from the app telling him that it is the right moment to head towards his destination. When he arrives he finds no queue outside the store, and, by simply scanning his QR code, he's able to enter and exit the shop in a safe and controlled environment.
    Tommy is a student and he’s preparing several exams for the upcoming session.
    He has just now noticed his fridge is empty and he decides to go grocery shopping. Since he needs to study hard and can't lose any time, he wants to go to the nearest supermarket. He takes out his phone and opens CLup. He immediately looks up for the store he wants to go to and in a few clicks he has enqueued himself! 
    Now it's time for Tommy to wait in the comfort of his house until he recives the notification from the app telling him that it is the right moment to head towards his destination. When he arrives he will find no queue outside the store: all time saved up for his studies!


    \item[Customer lines up]
    %As every monday, even this monday Mr. ZZZ goes to the nearby supermarket to gather all he needs for the week. He is used to wait for his turn standing outside in a queue and for someone of his age it's not a simple task. Approaching the entrance of the store he is really surprised noticing that there is no queue, and, rushing inside, he sees a new installation blocking his way: a turnstile and a ticket machine next to it. He reads all the new rules to follow and immediately after he clicked a few button on the new machine, he receives a ticket with the time he needs to wait to enter printed. Now Mr.ZZZ can safely wait his turn while sitting on the benches of the nearby park. When his time will come he'll be able to enter and exit the store scanning the ticket at the turnstile.
    Mrs Sunny would introduce herself as the proud grandmother of three beautiful kids.   
    Every monday, her grandchildren would come visit her, so  Mrs. Sunny goes to the nearby supermarket to gather all she needs. She used to wait for her turn standing outside in a queue and for someone of her age it's no simple task to stand on her feet for so long. Approaching the entrance of the store she is really surprised as she notices that there is no queue, and, rushing inside, she notices a new installation blocking her way: a turnstile and a ticket machine next to it. She reads all the new rules to follow and, immediately after she clicks a few buttons on the new machine, she receives a ticket that tells her how long she needs to wait to enter. Now Mrs. Sunny can safely wait his turn while sitting on the bench of the nearby park. 


    \item[User books a visit]
    %YYY has a really strict and rigid work schedule and lately he has found himself unable to go grocery shopping, because of long queues and closing hours of supermarkets. Luckily for him, one of his co-worker tells him that CLup app has a "book a visit" features to better help people manage their time. YYY decides to try it and, after downloading the app and signing up, he searches the store he needs to go to, checks the available reservations, and books the one right after work. The very next day at 7 p.m., when everybody is leaving the office, YYY receives a remainder from the CLup application about his reservation: he will no longer waste a minute in queues neither he'll be left without being able to do the shopping.
    Mr. Zehng  has a really strict and rigid work schedule and lately he has found himself unable to go grocery shopping, because of long queues and closing hours of supermarkets. Luckily for him, one of his co-worker tells him that CLup app has a "book a visit" features to better help people manage their time. Mr. Zehng decides to try it and, after downloading the app and signing up, he looks up for the store he needs to go to, checks the available reservations, and books the one right for him. The very next day at 7 p.m., when everybody is leaving the office, Mr. Zehng receives a reminder from CLup about his reservation: he will no longer waste a minute in queues neither he'll be left without being able to shop.

    \item[Manager registers a shop]
    %During quarantine shops are subjected to strict rule in order to guarantee health safety to their clients. WWW has dealt with a lot of legal problem during the first months of the quarantine, because of how difficult it can be to regulate and monitor traffic through all of his shops. This time he has decided to turn to CLup to precisely regulate the flow of customers and to avoid any problem with crowd forming inside or outside his shops. First of all he visited CLup website and by registering as manager... ?
    Mr. Hochikawa is the owner of a chain of supermarkets.
    During quarantine, shops are subjected to strict rules in order to guarantee health safety to their clients. Indeed, Mr. Hochikawa has been facing a lot of difficulties lately because of how hard it can be to regulate and monitor traffic through all of his shops. Now he has decided to turn to CLup to precisely regulate the flow of customers and to avoid any problem of people crowding up inside or outside his shops. After visiting the CLup website and completing the regitration, he can insert in the sytem all of hi hops, and now it has become just a question of a click to monitor the ones he wants whenever he wants to. Mr. Hochikawa’s life is so much easier now, thanks to Clup.

    \item[Manager updates shop's info]
    %The pandemic is unpredictable and therefore sometimes WWW finds himself having to close his supermarket, or changing some parameters because of new laws. CLup helps him customize everything he needs to, in fact in a few clicks from the app a shop manager can update all the shop's info. For example when… ?
    Miss Lebon owns a little grocery shop downtown and lately she had an hard time keeping up with the continuous changes of the governamental rules and to keep informed her employes and her clients about all of the necessary and sudden changes about the shop schedule or about how many people are allowed in the shop at a time. If in the past times this has been a very effortful and costly task for her it has now become much easier to do it thanks to the Clup updating functions, which manage to keep everybody informed with almost no effort! 
\end{description}

These scenarios presented highlight the main functionalities of CLup, however the application provides severals more. 

\subsubsection{Critical aspects}
\label{subsubsect:criticalaspects}

We are now going to expose how we decided to manage and solve some of the critical aspects of the system.

 Since all of the major functionalties of our application are data-related, in order to guarantee the effectiveness of our services, our application must work with precise and complete data. Three main features deserve particular focus:
\begin{itemize}[topsep=0pt]
    \item \textit{retrieve correct data from the people that interact with the system};
    \item \textit{work with data to deliver a good service, especially regarding the estimations};
    \item \textit{make the data accessible to the entire population};
\end{itemize}

The service we intend to deliver can \textit{retrive all of the fundamental data} at a software level, exception made for the matter of keeping track of the entrance and exit of the people from the shops integrated in our system. The solution to \textbf{regulate entrances and exits} is to have hardwares installed in the shops, consisting of a \textit{QR-code scanner} and a \textit{turnstile}. Indeed, when a user needs to enter and to exit a shop, they'll simply need to scan their QR-code and our system will keep track whever they are inside or outside of the shop. Furthermore, this mechanism will also help shop managers to prevent non authorized accesses to the store, so that no people will be able to enter a shop unless it's their turn.

To \textbf{reach the entire population of possible clients} we'll provide access to our system via website and mobile application, reaching this way the majority of the population. Furthermore we intend to install another hardware component, a \textit{totem}, outside of the shops, which will grant access to our system also from the spot.

\textbf{Making good estimations} is needed to deliver to the end users correct informations about the status of the shops, and, thus, even if a margin of error will be admitted, it is important for the user-expirience sake. The more we have reliable, anticipated and precise information about users' movements, the more our estimations we'll be correct and the general user-experience pleasant, but since the data we are working with are coming from real people, we can't have control over them. 

The best example of an estimation problem concerns queue management: without having control over people's actions, we cannot ensure precision regarding queue times estimations, because there may be sudden shifts of turns. The most important thing we can do is to build a good \textit{algorithm} to handle the queues, imagining that in most scenarios it will have no problem and it will be efficient and accurate, but in some edge cases it may not be able to organize the the enqueuements and booked visits in a perfectly no-time-waste way, therefore not optimizing the influx of people in the shops. To minimize the risk and to make the best estimations we intend to provide to the user the capability of helping the system with all of the possible information, such as, for example, how long their permancence in the shop will last, or if they are intentioned to \textit{cancel} reservations or enqueuements. We'll provide further details about the agorithm we intend to use in the Design Document.

Another aspect we want to highlight is the \textbf{tickets handling management system} for booking visits and enqueuing users.

Even if \textit{booking a visit} and \textit{enqueuing} seems to be two totally different functions to the end user, by the system perspective they are pretty much the same thing: a user who books a visit can be managed as if they were enqueuing for a specific hour and day, and a user who decides to enqueue themself as if they were booking a visit as soon as possible. 

Having specified these similarities, there are still some differences. A \textit{queue} can be seen as a list that keeps track of users' turns to enter a shop and it is not a static structure, but it evolves through time or because of particular events, such as users leaving queues or canceling booked visits or staying in shops more time than what they specified, and others. Our system will constantly keep the queues updated and will notify the interested users of every changes. 

Analyzing the behavior of the queue the only real difference between visit-tickets and queue-tickets comes to the surface: visit-tickets cannot shift in time, they are scheduled, but queue-tickets can.

So once a user dequeues themselves or books a visit the system generates a ticket and manages it according to what said above. Furthermore the system, in order to keep track of the condition of the tickets, follows the state diagram shown in the picture [\ref{fig:ticketlifecycle}]. A ticket can be in one of the following states:
\begin{itemize}[topsep=0pt]
    \item \textbf{valid ticket}: when a ticket is created it is always \textit{valid}, and that means that is waiting to be used to enter a shop.
    \item \textbf{in use ticket}: a ticket is \textit{in use} from the time it is used to enter the shop, to the time it is used to exit the shop.
    \item \textbf{used ticket}: a \textit{used} ticket is a ticket that has been successfully used to exit a shop. 
    \item \textbf{expired ticket}: \textit{expired} ticket is a ticket that can't be used to enter a shop because the turn they were valid for has passed. This state could have been avoided by integrating it in the \textit{used} state, but in order to have a better understanding of the life cycle of tickets we included it in here. 
\end{itemize}

\begin{figure}[h!]
    \centering
    \includegraphics[height=6cm]{Images/statediagrams/ticketlifecycle.png}
    \caption{\label{fig:ticketlifecycle}State diagram: ticket's life cycle}
\end{figure}

\FloatBarrier

\subsubsection{Product perspective}
\label{subsubsect:productperspective}

CLup services will not be integrated with any other system, it will be implemented from scratch and in a three-tier architecture following a distributed Model-View-Controller design pattern.

The \textit{Server-side} will host the entire logic of the application: the \textit{controller}. The \textit{Client-side} will take care of the presentation. Last, but not least, a dedicated server will host the database in which the system will store all of the data.

\subsection{Goals}
\label{subsect:goals}

The following tables show in details all of the main goals that our application meets. 

We managed to divide them in two main categories: goals related to the \textit{shop owners/managers} and goals related to the \textit{clients/users}. Also, we have assigned a unique identifier to each one of them in order to be able to refer to them later in the document.

%TODO: Allow users to report some wrong informations about a shop (così noi inviamo una notifica ai manager di controllare che magari han sbagliato a scrivere qualcosa). Non so se metterlo inceramente

\begin{table}[h!]
    \centering
    \begin{tabular}{@{}P{0.1\textwidth}P{0.80\textwidth}@{}}
        \multicolumn{2}{c}{\textbf{Shop owners}} \\
        \toprule
        \textbf{Identifier}& \textbf{Goal}\\
        \midrule
        \textbf{G1}        & Allow manager to sign on the system\\
        \textbf{G2}        & Allow a manager to sign in the system\\
        \textbf{G3}        & Allow a manager to register their store/stores on the system\\
        $\;\;$    G3.1  & Allow a manager to register basic info about shops\\
        $\;\;$    G3.2  & Allow a manager to divide their stores in areas\\
        $\;\;$    G3.3  & Allow a manager to register the items in the areas\\
        \textbf{G4}        & Allow a manager to update their shops informations and settings\\
        \textbf{G5}        & Allow a manager to check the general status and the statistics of their shops\\
        \textbf{G6}        & Allow a manager to cancel a previously booked shopping session for a user\\
    \end{tabular}
\caption{Shop owner's goals}
\label{table:shopownersgoals}
\end{table}

\begin{table}[h!]
    \centering
    \begin{tabular}{@{}P{0.1\textwidth}P{0.80\textwidth}@{}}
        \multicolumn{2}{c}{\textbf{Clients}} \\
        \toprule
        \textbf{Identifier}& \textbf{Goal}\\
        \midrule
        \textbf{G7}        & Allow a user to sign on the system\\
        \textbf{G8}       & Allow a user to sign in the system\\
        \textbf{G9}       & Allow a user to join a virtual queue\\
        \textbf{G10}        & Allow a customer to join a virtual queue from nearby the shop\footnotemark[1]\\
        \textbf{G11}       & Allow a user to book a shopping session at a grocery store\\
        $\;\;$    G11.1 & Allow a user to select the duration of their shopping session\\
        $\;\;$    G11.1 & Allow a user to select the time for their shopping session from a list of available options\\ 	
        $\;\;$    G11.2 & Allow a user to select categories of items they are willing to buy\\
        \textbf{G12}       & Allow a user to retrieve information about their previously booked visits\\
        \textbf{G13}       & Allow a user to retrieve information about current enqueuements\\
        \textbf{G14}       & Allow a user to retrieve information about shops\\
        \textbf{G15}       & Allow a users and customers to enter and exit stores with QR-codes\\
        \textbf{G16}       & Allow a user to exit a previously joined queue \\
        \textbf{G17}       & Allow customers to line up\footnotemark[2]\\
        \textbf{G18}       & Allow a customer to cancel a previously booked visit \\
        \textbf{G19}       & Allow a user to cancel a previously booked visit\footnotemark[1] \\

        \bottomrule
    \end{tabular}
\caption{Client's goals}
\label{table:clientsgoals}
\end{table}
\footnotetext[1]{with the facility of a hardware support, it will be better specificed later on this document.}
\footnotetext[2]{to the customer's eyes it seems like an enqueuement, but from the system perspective the entire process is handled as a booked visits, it will be better specified later on this document. TODO: is it really specified later in this document? check.}

\FloatBarrier

\subsection{Domain Assumptions}
\label{subsect:domainassumptions}

The following table lists all the assumptions over which we cannot have control, but we assume as verified, and, as with the goals in the section above, we provide them with a unique identifier to keep track of them in the document.

\begin{table}[h!]
    \centering
    \begin{tabular}{@{}P{0.1\textwidth}P{0.80\textwidth}@{}}
        \toprule
        \textbf{identifier}& \textbf{Domain Assumption}\\
        \midrule
        \textbf{D1}        & Data provided from GPS is valid and accurate with regard to the real position of the users with a less than $50$ meters radius \footnote{}\\
        \textbf{D2}        & Google Maps' paths calculator makes correct time estimations\\
        \textbf{D3}        & A person cannot get in the store without having scanned a valid QR-code\\
        \textbf{D4}        & A person cannot get out of the store without having scanned a valid QR-code\\
        \textbf{D5}        & Each store registered on the system is provided with the necessary functioning hardware\\
        \textbf{D6}        & Data provided from managers is legit\\
        $\;\;$D6.1         & Managers provide the maximum number of people allowed in the store by the law\\
        $\;\;$D6.2         & Managers will provide the correct address of their store\\
        $\;\;$D6.3         & Managers will provide the correct schedule of their store\\
        $\;\;$D6.4         & Managers will provide the correct categories of items they sell\\
        $\;\;$D6.5         & Managers will correctly map items with the respective category\\
        \textbf{D7}        & Every category of a shop is matched with an area non overlapping with any other category area\\
        \textbf{D8}        & A customer will take the shortest path to the store\\ %(TODO: non ricordo se alla fine volevamo tenerlo o meno\dots)
        \textbf{D9}        & Customers will stay in the store approximately the time they have claimed they would\\ %(TODO: non ricordo se alla fine volevamo tenerlo o meno\dots)
        \textbf{D10}       & Customers will shop the items they have claimed they would\\
        \textbf{D11}       & Phone numbers are unique\\
        \textbf{D12}       & Emails are unique\\
        \textbf{D13}       & Notification sent to the users, may them be managers or clients, will be received and comprehended by them\\
        \textbf{D14}       & People have no malicious behavior against the app\\
        \textbf{D15}       & Managers' emails are correctly validated by PEC system ( or any secure email system provided by the country of pertinence)\\
        \textbf{D16}       & Users are aware of the process required to scan a QR-code\\
        \textbf{D17}       & One QR-code scanned means that one and only one person enters the store\\
        \bottomrule
    \end{tabular}
\caption{Domain assumptions}
\label{table:domainassumptions}
\end{table}
\footnotetext{We do not need a high level precision of users' positions in order to have a funcitoning system, but, obviously, the more precise the better the user expirience.}

\FloatBarrier

\subsection{Constraints}
\label{subsect:contraints}

\subsubsection{Regulatory policies}
\label{subsubsect:regulatorypolicies}

As CLup is an application that deals with real people data, it must follow strict rules and laws regarding the privacy of its users.

Here we show a list of mandatory aspects to consider in order to correctly release an application like CLup:
\begin{itemize}
    \item protect application and users' interests with Terms ans Conditions;
    \item ask users permission to retrieve and use GPS data;
    \item regulate cookie management;
    \item telephone numbers and emails will not be used for commercial purpose;
    \item all data about user's booked visits or enqueued clients accessible for statistics and monitoring purposes is stored in an anonymous way.
\end{itemize}

\subsubsection{Hardware limitations}
\label{subsubsect:hardwarelimitations}

In order to reach all kind of users and customers, our application wants to be the most accessible as possible. To achive this goal we intend to release various version of the application for different devices:

\begin{itemize}
    \item \textbf{Mobile app}: Requires a smartphone with internet connection, either Android or Ios. It is not necessary to have GPS to run and use the application, but only to have access to some features.
    \item \textbf{Web app}: Requires a modern browser with javascript enabled and with an internet connection. The capability of retrieving user location is not necessary, but in order to have access to some features is mandatory.
    \item \textbf{Shops hardware}: Our intent is to have shop owners install hardware facilities that comprehend a totem, a QR-code scanner and an electronic turnstile. The totem will have internet connection and run a special version of the web application. The QR-code scanner and electronic turnstile will work togheter to allow access to the stores. 
\end{itemize}

\subsubsection{Interfaces to other applications}
\label{subsubsect:interfacestootherappications}

Our application will rely on external services to handle some features:

\begin{itemize}
    \item Our sytem will interact wit Google Maps API.
    \item Interface with SMS gateway provider via standard SMS rest APIs.
    \item Interface with database. %TODO: dovremmo magari dire qualcosa di più
\end{itemize}

\subsection{Use cases}
\label{subsect:usecases} 

In this section we want to present all the major activities concerning the interaction between the application and the actors.

We will present a table for each use case and, for the most important ones, a list of sequence diagrams that will cover all the interaction between the actors and the system in a graphical way. For better clarity in the exposure and navigation throught the document we present an organized index (table [\ref{table:usecaseindex}]) of all the use cases with the relatives links to tables and, if present, to sequence diagrams.
\begin{table}[h!]
    \centering
    \begin{tabular}{@{}P{0.57\textwidth}P{0.20\textwidth}P{0.20\textwidth}@{}}
        & \textbf{Description table} & \textbf{Sequence Diagram}\\
        \toprule
        \textbf{1. Customer registration} & tab.\ref{table:customerregitration} & \\
        \textbf{2. shop owner registration} & tab.\ref{table:shopownerregistration} & \\
        \textbf{3. User logs in} & tab.\ref{table:login} & \\
        \textbf{4. Manager logs in} & tab.\ref{table:login} & \\
        \textbf{5. User searches shop}& tab.\ref{table:usersearchesshop} & fig.\ref{fig:usersearchesshop}\\
        \textbf{6. User gets more information} & tab.\ref{table:usergetmoreinformation}& \\
        \textbf{7. User lines up} & tab.\ref{table:userlinesup} & fig.\ref{fig:userlinesup}\\
        \textbf{8. User books a visit} & tab.\ref{table:userbooksavisit} & fig.\ref{fig:userbookavisit}\\
        \textbf{9. Customer lines up} & tab.\ref{table:customerlinesup}& fig.\ref{fig:customerlinesup}\\
        \textbf{10. User/customer enters shop} & tab.\ref{table:entershop} & \\
        \textbf{11. User/customer exits shop} & tab.\ref{table:exitshop} & \\
        \textbf{12. Manager registers new shop} & tab.\ref{table:managerregisternewshop} & \\
        \textbf{13. Manager updates shop's information and settings} & tab.\ref{table:managerupdatesshopinformation} & \\
        \textbf{14. Manager checks shop status} & tab.\ref{table:managerchecksshopstatus} & \\
        \textbf{15. User cancels previously booked visit} & tab.\ref{table:usercancelspreviouslybookedvisit} & \\
        \textbf{16. Customer cancels previously booked visit} & tab.\ref{table:customercnacelspreviousenqueuement} & \\
        \textbf{17. Manager cancels customer's previously booked visit} & tab.\ref{table:managercancelscustomerspreviouslybookedvisit} & \\
        \textbf{18. User cancels previous enqueuement} & tab.\ref{table:usercnacelspreviousenquement} & \\
        \bottomrule
    \end{tabular}
\caption{All use cases}
\label{table:usecaseindex}
\end{table}

\FloatBarrier

\subsubsection{Use cases tables}

\begin{table}[h!]
    \centering
    \begin{tabular}{@{}P{0.2\textwidth}P{0.70\textwidth}@{}}
        \toprule
        \textbf{Name}                 & Customer registration\\
        \midrule
        \textbf{Actors}               & customer\\
        \textbf{Entry condition}      & the customer is already on the front page of the web or mobile application\\
        \textbf{Flow of events}       & 
        \begin{enumerate}[nolistsep, leftmargin=*]
            \item the customer selects the sign up option
            \item the customer fills in all of the mandatory fields and then confirms
            \item the system sends an SMS with a code to the customer's previously specified phone number and waits for an a answer
            \item the customer validates his phone number by inserting the code
            \item the systems sends an email with a code to the customer's previously specified address and waits for an a answer
            \item the customer validates his email address by inserting the code
            \item the system create a new account and redirects the new user to his personal home page
        \end{enumerate} \\
        \textbf{Exit condition}       & the customer successfully ends the registration process and becomes a new user\\
        \textbf{Exceptions}           
        & - timer expiration\\
        & - wrong codes\\
        & - email or phone number already registered or not valid\\
        & - the customer insterts not valid informations in one or more mandatory fields\\
        \bottomrule
    \end{tabular}
\caption{Customer registration}
\label{table:customerregitration}
\end{table}

\begin{table}[h!]
    \centering
    \begin{tabular}{@{}P{0.2\textwidth}P{0.70\textwidth}@{}}
        \toprule
        \textbf{Name}                 & Shop owner registration\\
        \midrule
        \textbf{Actors}               & shop owner\\
        \textbf{Entry condition}      & the shop owner is already on the front page of the web or mobile application\\
        \textbf{Flow of events}       & 
        \begin{enumerate}[nolistsep, leftmargin=*]
            \item the shop owner selects the sign up option
            \item the shop owner specifies he wants to sign up as a manager
            \item the customer fills in all of the mandatory fields\footnote{one of the mandatory fields is the PEC email} and then confirms
            \item the system sends an SMS with a code to the shop owner's previously specified phone number and waits for an a answer
            \item the shop owner validates his phone number by inserting the code
            \item the systems sends an email with a code to the shop owner's previously specified address and waits for an a answer
            \item the shop owner validates his email address by inserting the code
            \item the systems validates the email address via PEC system
            \item the system create a new account and redirects the new manager to his personal home page
        \end{enumerate} \\
        \textbf{Exit condition}       & the shop owner successfully ends the registration process and become a new manager\\
        \textbf{Exceptions}           
        & - timer expiration\\
        & - wrong codes\\
        & - email or phone number already registered or not valid\\
        & - PEC email not verified\\
        & - the customer insterts not valid informations in one or more mandatory fields\\
        \bottomrule
    \end{tabular}
\caption{Shop owner registration}
\label{table:shopownerregistration}
\end{table}

\begin{table}[h!]
    \centering
    \begin{tabular}{@{}P{0.2\textwidth}P{0.70\textwidth}@{}}
        \toprule
        \textbf{Name}                 & User/Manager log in\\
        \midrule
        \textbf{Actors}               & user or manager\\
        \textbf{Entry condition}      & the actor is already on the front page of the web or mobile application\\
        \textbf{Flow of events}       & 
        \begin{enumerate}[nolistsep, leftmargin=*]
            \item the actor selects the sing in option
            \item the actor compiles the email and password fields and confirms by clicking on the log in button 
            \item the system redirects the actors to the respective home pages
        \end{enumerate} \\
        \textbf{Exit condition}       & the actors are successfully redirected to their respective personal home page\\
        \textbf{Exceptions}           
        & - email or password aren’t correct \\
        \bottomrule
    \end{tabular}
\caption{User/Manager log in}
\label{table:login}
\end{table}

\begin{table}[h!]
    \centering
    \begin{tabular}{@{}P{0.2\textwidth}P{0.70\textwidth}@{}}
        \toprule
        \textbf{Name}                 & User searches a shop\\
        \midrule
        \textbf{Actors}               & user\\
        \textbf{Entry condition}      & user logged in and on his web or mobile personal home page\\
        \textbf{Flow of events}       & 
        \begin{enumerate}[nolistsep, leftmargin=*]
            \item user selects the search a shop option
            \item the system gets the user's position using GPS and after retrieving the shops adhering to CLups near him, shows them in a map and, also, in a list
            \item the user either types the name of the shop they want to find in the search bar or navigates the map or scrolls the list
            \item the user selects a shop
            \item the application retrieves all the necessary information about the selected shop's status and information, displays them to the user with the list of possible action the user can take: book a visit or enqueue or get more info
        \end{enumerate} \\
        \textbf{Exit condition}       & the user successfully finds the shop he was looking for\\
        \textbf{Exceptions}           
        & - the shop searched doesn't exists or isn't adhering to CLup\\
        & - GPS is off\\
        \textbf{Notes} & an user searches for a shop in order to get general informations or check its status or to enqueue himself or to book a visit\\
        \bottomrule
    \end{tabular}
\caption{User searches a shop}
\label{table:usersearchesshop}
\end{table}

\begin{table}[h!]
    \centering
    \begin{tabular}{@{}P{0.2\textwidth}P{0.70\textwidth}@{}}
        \toprule
        \textbf{Name}                 & User gets more information\\
        \midrule
        \textbf{Actors}               & user\\
        \textbf{Entry condition}      & the user is successfully at the exit condition of "user searches a shop" use case (table \ref{table:usersearchesshop})\\
        \textbf{Flow of events}       & 
        \begin{enumerate}[nolistsep, leftmargin=*]
            \item the user selects the get more info option
            \item the system retrieves further details about the shop and displays them to the user
            \item the user visualizes the informations
        \end{enumerate} \\
        \textbf{Exit condition}       & the user can visualize the information needed\\
        \bottomrule
    \end{tabular}
\caption{User gets more information}
\label{table:usergetmoreinformation}
\end{table}

\begin{table}[h!]
    \centering
    \begin{tabular}{@{}P{0.2\textwidth}P{0.70\textwidth}@{}}
        \toprule
        \textbf{Name}                 & User lines up\\
        \midrule
        \textbf{Actors}               & user\\
        \textbf{Entry condition}      & the user is successfully at the exit condition of "user searches a shop" use case (table \ref{table:usersearchesshop})\\
        \textbf{Flow of events}       & 
        \begin{enumerate}[nolistsep, leftmargin=*]
            \item the user selects the line up option
            \item the user specifies the estimated time he thinks he will spend at the shop
            \item optionally the user either activates GPS and specifies by what mean they are going to reach the shop or specifies how much time he needs to reach the shop
            \item if the store has divided his store in categories, the user can optionally specify what categories he is interested in
            \item the system adds the user to the queue and displays a constantly updated page with all the relevant information about the enqueuement
        \end{enumerate} \\
        \textbf{Exit condition}       & the user is correctly enqueued and needs to wait for his turn\\
        \textbf{Exceptions}           
        & - the enqueuement is not possible\\ %some weird and really really edge cases of simmultaneus enqueuement
        & - GPS is not active\\
        \textbf{Notes} & the page with all the enqueuement information can be also reached by the user at any moment from his personal home page when he is currently enqueued\\
        \bottomrule
    \end{tabular}
\caption{User lines up}
\label{table:userlinesup}
\end{table}

\begin{table}[h!]
    \centering
    \begin{tabular}{@{}P{0.2\textwidth}P{0.70\textwidth}@{}}
        \toprule
        \textbf{Name}                 & User books a visit\\
        \midrule
        \textbf{Actors}               & user\\
        \textbf{Entry condition}      & the user is successfully at the exit condition of "user searches a shop" use case (table \ref{table:usersearchesshop})\\
        \textbf{Flow of events}       & 
        \begin{enumerate}[nolistsep, leftmargin=*]
            \item the user selects the book a visit option
            \item the user specifies the estimated time he thinks he will spend at the shop
            \item if the store has divided his store in categories, the user can optionally specify what categories he is interested in
            \item the system retrieves the possible slots and displays the options to the user
            \item the user chooses the slots he prefers and confirms
            \item the system registers the visit, sends confirmation to the user and redirects him to a page with all the relevant infromation about the booked visit
        \end{enumerate} \\
        \textbf{Exit condition}       & the user has correctly booked a visit\\
        \textbf{Exceptions}           
        & - the reservation is not possible\\
        & - the user can't find any available slot\\
        \textbf{Notes} 
        & -the order of the various information requested to the user is important in order to let the system decide what possible slots are available\\
        & -The page with all relevant information about the visit is easily accessible: from the user's personal home page there's an option to check all the reservation, this option leads to a page with all the reservations and by selecting one the user can reach the said page\\
        \bottomrule
    \end{tabular}
\caption{User books a visit}
\label{table:userbooksavisit}
\end{table}

\begin{table}[h!]
    \centering
    \begin{tabular}{@{}P{0.2\textwidth}P{0.70\textwidth}@{}}
        \toprule
        \textbf{Name}                 & Customer lines up\\
        \midrule
        \textbf{Actors}               & customer\\
        \textbf{Entry condition}      & the customer is at the entrance of the hop and have read and understood the functioning of the hardware\\
        \textbf{Flow of events}       & 
        \begin{enumerate}[nolistsep, leftmargin=*]
            \item the customer visualize information about the status of the shop
            \item the customer select the enqueue option on the totem
            \item the customer specifies the estimated time he thinks he will spend at the shop
            \item the system books a visit for the customer at the first available slot
            \item the totem retrieves the visit information and prints a ticket with them
            \item the customer takes the ticket
        \end{enumerate} \\
        \textbf{Exit condition}       & the cusotmer succesfully retrieves his ticket and has a booked visit registered\\
        \textbf{Exceptions}           
        & - enqueuement is not possible\\
        & - hardaware is not functioning properly\\
        \bottomrule
    \end{tabular}
\caption{Customer lines up}
\label{table:customerlinesup}
\end{table}

\begin{table}[h!]
    \centering
    \begin{tabular}{@{}P{0.2\textwidth}P{0.70\textwidth}@{}}
        \toprule
        \textbf{Name}                 & User/customer enters the shop\\
        \midrule
        \textbf{Actors}               & user, customer\\
        \textbf{Entry condition}      & have previously enqueued or have previously booked a visit\\
        \textbf{Flow of events}       & 
        \begin{enumerate}[nolistsep, leftmargin=*]
            \item the actor arrives at the shop
            \item the actor correctly places the QR-code on the QR-code scanner facility at the entrance of the shop
            \item the system analyze the QR-code
            \item the system unlocks the turnstile
            \item the actor enters the shop
        \end{enumerate} \\
        \textbf{Exit condition}       & the user/customer get inside the shop\\
        \textbf{Exceptions}           
        & - the QR code is not valid\\
        & - the hardware is not functioning properly\\
        \bottomrule
    \end{tabular}
\caption{User/customer enters a shop}
\label{table:entershop}
\end{table}

\begin{table}[h!]
    \centering
    \begin{tabular}{@{}P{0.2\textwidth}P{0.70\textwidth}@{}}
        \toprule
        \textbf{Name}                 & User/customer exits the shop\\
        \midrule
        \textbf{Actors}               & user, customer\\
        \textbf{Entry condition}      & have previously entered the store with a QR code\\
        \textbf{Flow of events}       & 
        \begin{enumerate}[nolistsep, leftmargin=*]
            \item the actor arrives at the exit of the shop
            \item the actor correctly places the QR-code on the QR-code scanner facility at the exit of the shop
            \item the system analyze the QR-code
            \item the system unlocks the turnstile
            \item the actor exit the shop
        \end{enumerate} \\
        \textbf{Exit condition}       & the user/customer get out of the shop\\
        \textbf{Exceptions}           
        & - the QR code is not valid\\
        & - the hardware is not functioning properly\\
        \bottomrule
    \end{tabular}
\caption{User/customer exits the shop}
\label{table:exitshop}
\end{table}

\begin{table}[h!]
    \centering
    \begin{tabular}{@{}P{0.2\textwidth}P{0.70\textwidth}@{}}
        \toprule
        \textbf{Name}                 & Manager registers new shop\\
        \midrule
        \textbf{Actors}               & manager logged in and on his web or mobile personal home page\\
        \textbf{Entry condition}      & \\
        \textbf{Flow of events}       & 
        \begin{enumerate}[nolistsep, leftmargin=*]
            \item the manager selects the option to register a new shop, a page pops up with a form to be filled
            \item  the manager fills in the mandatory fields (position of the shop, schedule of the shop, opening days, name of the shop, maximum number of people allowed in the shop, optionally the items sold in the shop divided in categories or, if preferred, just the categories)
            \item the manager submit the data inserted
            \item the system verifies the data inserted
            \item the system correctly registers the shop
            \item the system sends confirmation to the manager
        \end{enumerate} \\
        \textbf{Exit condition}       & the shop has been correctly registered\\
        \textbf{Exceptions}           
        & - any of the data inserted in the mandatory fields is invalid\\
        \bottomrule
    \end{tabular}
\caption{Manager registers new shop}
\label{table:managerregisternewshop}
\end{table}

\begin{table}[h!]
    \centering
    \begin{tabular}{@{}P{0.2\textwidth}P{0.70\textwidth}@{}}
        \toprule
        \textbf{Name}                 & Manager updates shop's information and settings\\
        \midrule
        \textbf{Actors}               & manager\\
        \textbf{Entry condition}      & manager logged in and on his web or mobile personal home page\\
        \textbf{Flow of events}       & 
        \begin{enumerate}[nolistsep, leftmargin=*]
            \item the manager selects the update shop info option
            \item the system redirects the manager to a dedicated page showing all the editable informations and settings
            \item the manager selects the desired information or setting to updadte
            \item the manager modifies the selected information or settings
            \item the manager confirms the update
            \item the system updates the shop informations
            \item the system sends confirmation to the manager
        \end{enumerate} \\
        \textbf{Exit condition}       & the shop informations or settings are correctly modified\\
        \textbf{Exceptions}           
        & - some new information or setting inserted from the manager is invalid\\
        \bottomrule
    \end{tabular}
\caption{Manager updates shop information}
\label{table:managerupdatesshopinformation}
\end{table}

\begin{table}[h!]
    \centering
    \begin{tabular}{@{}P{0.2\textwidth}P{0.70\textwidth}@{}}
        \toprule
        \textbf{Name}                 & Manager checks shop status\\
        \midrule
        \textbf{Actors}               & manager\\
        \textbf{Entry condition}      & manager logged in and on his web or mobile personal home page\\
        \textbf{Flow of events}       & 
        \begin{enumerate}[nolistsep, leftmargin=*]
            \item the manager selects the check shop status option
            \item the system retrieves all the informations about the shop and displays them to the manager (number of people actually inside the shop, number of people enqueued, estimated time of enqueuement, upcoming booked visits, etc)
            \item the manager correctly visualizes such informations
        \end{enumerate} \\
        \textbf{Exit condition}       & the manager correctly visualizes the shop status\\
        \bottomrule
    \end{tabular}
\caption{Manager checks shop status}
\label{table:managerchecksshopstatus}
\end{table}

\begin{table}[h!]
    \centering
    \begin{tabular}{@{}P{0.2\textwidth}P{0.70\textwidth}@{}}
        \toprule
        \textbf{Name}                 & User cancels previously booked visit\\
        \midrule
        \textbf{Actors}               & user\\
        \textbf{Entry condition}      & user logged in, on his web or mobile personal home page\\
        \textbf{Flow of events}\footnote{the first two events represent the flow of events the user must follow to check his previosuly booked visits}       & 
        \begin{enumerate}[nolistsep, leftmargin=*]
            \item the user selects the check previously booked visits option
            \item the system retrieves the user's previously booked visits
            \item the user selects one of his previously booked visits
            \item the user selects the cancel prenotation option
            \item the system cancels the booked visit and notifies the user of success
        \end{enumerate} \\
        \textbf{Exit condition}       & the user successfully canceles the visit\\
        \textbf{Exceptions}           
        & - user has no previously booked visits\\
        \bottomrule
    \end{tabular}
\caption{User cancels previously booked visit}
\label{table:usercancelspreviouslybookedvisit}
\end{table}

\begin{table}[h!]
    \centering
    \begin{tabular}{@{}P{0.2\textwidth}P{0.70\textwidth}@{}}
        \toprule
        \textbf{Name}                 & Customer cancels previous enqueuement TODO: come detto per i requirements: decidiamo che termini usare, enqueuement o booked visit...\\
        \midrule
        \textbf{Actors}               & customer\\
        \textbf{Entry condition}      & customer infront of the a shop totem that is displaying his home page\\
        \textbf{Flow of events}       & 
        \begin{enumerate}[nolistsep, leftmargin=*]
            \item the customer selects the cancel a booked visit option
            \item the user insterts the code of his ticket on the totem
            \item the system cancels the booked visit and notifies the customer of success
        \end{enumerate} \\
        \textbf{Exit condition}       & the customer successfully cancels his enqueument\\
        \textbf{Exceptions}           
        & - the customer has no booked visit\\
        & - the customer uses the ticket from an already elapsed visit\\
        \bottomrule
    \end{tabular}
\caption{Customer cancels previously enqueuement}
\label{table:customercnacelspreviousenqueuement}
\end{table}

\begin{table}[h!]
    \centering
    \begin{tabular}{@{}P{0.2\textwidth}P{0.70\textwidth}@{}}
        \toprule
        \textbf{Name}                 & Manager cancels customer’s previously booked visit\\
        \midrule
        \textbf{Actors}               & manager\\
        \textbf{Entry condition}      & the manager is logged in and on his web or mobile personal home page\\
        \textbf{Flow of events}       & 
        \begin{enumerate}[nolistsep, leftmargin=*]
            \item manager selects the cancel customer's previously booked visit option 
            \item the system ask the manager to insert the code of the ticket to retrieve the desiderd booked visit
            \item the manager inserts the code
            \item the system cancel the booked visit and notifies the user of the success of the process
        \end{enumerate} \\
        \textbf{Exit condition}       & the booked visit is successfully canceled\\
        \textbf{Exceptions}           
        & -\\
        \bottomrule
    \end{tabular}
\caption{Manager cancels customer’s previously booked visit}
\label{table:managercancelscustomerspreviouslybookedvisit}
\end{table}

\begin{table}[h!]
    \centering
    \begin{tabular}{@{}P{0.2\textwidth}P{0.70\textwidth}@{}}
        \toprule
        \textbf{Name}                 & User cancels previous enqueuement\\
        \midrule
        \textbf{Entry condition}      & user logged in, on his web or mobile personal home page\\
        \textbf{Flow of events}\footnote{The first two steps represents the flow of events the user must follow to check his current enqueuement}       & 
        \begin{enumerate}[nolistsep, leftmargin=*]
            \item the user selects the virtual ticket in his home page
            \item the system retrieves and displays all the information about the ticket and the queue
            \item the user selects the dequeue button
            \item the system correctly dequeue the user and notifies the user of success
            \item the system notifies all the user that changed position in the queue
        \end{enumerate} \\
        \textbf{Exit condition}       & the user successfully cancels the visit\\
        \textbf{Exceptions}           
        & - the user is not enqueued\\
        \bottomrule
    \end{tabular}
\caption{User cancels previous enqueuement}
\label{table:usercnacelspreviousenquement}
\end{table}

\clearpage

\subsubsection{Use cases sequence diagrams}

\begin{figure}[h!]
    \centering
    \includegraphics[width=\textwidth]{Images/sequencediagrams/UsersearchesashopSD.png}
    \caption{\label{fig:usersearchesshop}User searches a shop - sequence diagram}
\end{figure}

\begin{figure}[h!]
    \centering
    \includegraphics[width=\textwidth]{Images/sequencediagrams/UserlinesupSD.png}
    \caption{\label{fig:userlinesup}User lines up - sequence diagram}
\end{figure}

\begin{figure}[h!]
    \centering
    \includegraphics[width=\textwidth]{Images/sequencediagrams/UserbooksavisitSD.png}
    \caption{\label{fig:userbookavisit}User books a visit - sequence diagram}
\end{figure}

\begin{figure}[h!]
    \centering
    \includegraphics[width=\textwidth]{Images/sequencediagrams/CustomerlinesupSD.png}
    \caption{\label{fig:customerlinesup}Customer lines up - sequence diagram}
\end{figure}

\FloatBarrier

\subsection{User interfaces}
\label{subsect:userinterfaces}

In this section we want to better clarify the \textit{interactions} between a \textit{regular user and CLup system}, through either the web application or the mobile app. 

The following diagram wants to connect all the user-related use cases listed previously, to give the reader an high-level understanding of how they are linked. By doing so the diagram shows also how the \textit{user interface} is structured, leaving space for mockups in the Design Document.

\begin{figure}[h!]
    \centering
    \includegraphics[width=\textwidth]{Images/statediagrams/userinterface.png}
    \caption{\label{fig:userinterfacestructure}{User interface structure - state diagram}}
\end{figure}

\FloatBarrier

\subsection{Application model}
\label{subsect:applicationstructure}

Here we present an early idea of an high-level abstraction of our application's structure. To be noticed is that it is mainly modeled with focus on the data and their interaction.

TODO: change file... aspettando LudoLe
\begin{figure}[h!]
    \centering
    \includegraphics[width=\textwidth]{Images/uml/todo.png}
    \caption{\label{fig:applicationmodel}{Application model - UML}}
\end{figure}