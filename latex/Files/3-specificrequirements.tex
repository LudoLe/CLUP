\section{Specific requirements}
\label{sect:Specificrequirements}

Here we include more details on all aspects in Section $2$ if they can be useful for the development team. 

The requirements are complete if they ensure satisfaction of the goals in the context of the domain properties (Analogy with program correctness: a Program P running on a particular Computer C is correct if it satisfies the Requirements R), also the goals adequately capture all the stakeholders’ needs, and the domain represents valid properties/assumptions about the world

\subsection{External interface requirements}
\label{subsect:ecternalinterfacerequirements}

\subsubsection{User interfaces}
\label{subsubsect:userinterfaces}

\subsubsection{Hardware interfaces}
\label{subsubsect:hardwareinterfaces}

\subsubsection{Software interfaces}
\label{subsubsect:softwareinterfaces}

\subsubsection{Communication interfaces}
\label{subsubsect:communicationinterfaces}

\subsection{Functional requirements}
\label{subsect:functionalrequirements}

Definition of use case diagrams, use cases and associated sequence/activity diagrams, and mapping on requirements.

Functional requirements: Describe the interactions between the system and its environment independent from implementation. Are the main goals the software to be has to fulfill.

Deriving requirements from goals: Must find the constraints on shared phenomena to be enforced by the machine to achieve the goals.

this section is organized by mode, user class, feature, etc. For example:

3.2 Functional Requirements

3.2.1 User Class 1

3.2.1.1 Functional Requirement 1.1

…

\subsection{Performance requirements}
\label{subsect:performancerequirements}

\subsection{Design constraints}
\label{subsect:designconstraints}

\subsubsection{Standards compliance}
\label{subsubsect:standardscompliance}

\subsubsection{Hardware limitations}
\label{subsubsect:hardwarelimitations}

\subsubsection{Any other constraint}
\label{subsubsect:anyotherconstraint}

\subsection{Software system attributes (Non functional requirements)}
\label{subsect:softwaresystemattributes}

Non functionals requirements ideas:
\begin{itemize}
    \item security issues: point/rating system to block people out. In first implementation we will not make it but we consider that a more official release will have this feature.
    \item enqueue with a friend features
    \item after a manager has sent a request to register a shop, he will receive a call from our team and they will organize everything that needs to be done... oppure email certificata e partita iva.. oppure boh, dobbiamo informarci
\end{itemize}

Nonfunctional requirements: User visible aspects of the system not directly related to functional behavior. 
Constraints on how functionality has to be provided to the end user.
Independent of the application domain, but the application domain determines their relevance and their prioritization.

Also called Quality of Service (QoS) attributes, here are some:
\begin{itemize}
    \item Extremely visible properties: Performance, Reliability, Scalability, Capacity, Accuracy, Accessibility, Availability.
    \item How the system works in unexpected/fault conditions: Robustness, Exception handling.
    \item Systems developed with different frameworks can work together at run time: Interoperability.
    \item Security issues: Integrity, Confidentiality.
\end{itemize}

\subsubsection{Reliability}
\label{subsubsect:reliability}

\subsubsection{Availability}
\label{subsubsect:availability}

\subsubsection{Security}
\label{subsubsect:security}

\subsubsection{Mantainability}
\label{subsubsect:mantainability}

\subsubsection{Scalability}
\label{subsubsect:scalability}

\subsubsection{Portability}
\label{subsubsect:portability}

\subsection{Other requirements}
\label{subsect:otherrequirements}

This section is not necessary.
