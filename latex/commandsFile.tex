\documentclass [11pt,twoside]{article}
\usepackage[utf8]{inputenc}
\usepackage[T1]{fontenc}

\usepackage[english]{babel}

%For LaTex syntax checking without bilding .pdf file
\usepackage{syntonly}
%\syntaxonly

%Page margins, header and footer positions
\usepackage{geometry}
 \geometry{
 a4paper,
 total={210mm,297mm},
 left=25mm,
 right=25mm,
 top=30mm,
 bottom=25mm,
 headsep=7mm}

\interfootnotelinepenalty=10000

%paragraph spacing
\setlength{\parskip}{1em}
%avoid the paragraph spacing to be applied to the table of content, the list of figures and the list of tables
\AtBeginDocument{\addtocontents{toc}{\protect\setlength{\parskip}{0pt}}}
\AtBeginDocument{\addtocontents{lof}{\protect\setlength{\parskip}{0pt}}}
\AtBeginDocument{\addtocontents{lot}{\protect\setlength{\parskip}{0pt}}}

%for nice tables
\usepackage{booktabs}
\usepackage{array}
\newcolumntype{P}[1]{>{\raggedright\arraybackslash}p{#1}}
\usepackage{longtable} %multi-page tables
\usepackage{multirow}

%handling floats
\usepackage{placeins}

%for placements of floats in the [H] here position
\usepackage{here}

%titles font size
\usepackage{titlesec}
\titleformat*{\section}{\huge\bfseries}
\titleformat*{\subsection}{\LARGE\bfseries}
\titleformat*{\subsubsection}{\Large\bfseries}
\titleformat*{\paragraph}{\large\bfseries}
\titleformat*{\subparagraph}{\large\bfseries}

%footnotes
\usepackage{footnote}

%To display filling dots in the TOC for all entries
\usepackage[titles]{tocloft}
\renewcommand{\cftsecleader}{\cftdotfill{\cftdotsep}}

%Define new header and footer style
\usepackage{fancyhdr}

\pagestyle{fancy}
\fancyhf{}
\lhead{\color{Gray}{\small{Travlendar+ project by YOUR NAMES}}}
\lfoot{\textcolor{Gray}{\small{Copyright © 2017, YOUR NAMES – All rights reserved}}}
\rfoot{\textcolor{Gray}{\thepage}}
\renewcommand{\headrulewidth}{0pt}

%PACKAGES
\usepackage{wasysym}
\usepackage{pifont}

\newcommand{\supported}{\ding{52}\xspace}
\newcommand{\unsupported}{\ding{55}\xspace}
\newcommand{\partsupported}{\textcolor{black!40}{\ding{52}}\xspace}
\newcommand{\lowsupported}{\textcolor{black!20}{\ding{52}}\xspace}
\newcommand{\unknowsupported}{\textbf{?}\xspace}

%Font: Times
\usepackage{times}
%Change monospaced font
\renewcommand{\ttdefault}{lmtt}

%tables
\usepackage{tabu}
\usepackage{tabularx}
\usepackage{ltablex}
\usepackage{longtable}
\usepackage{float} % To allow the use of H modifier in long tables

%landscape mode
\usepackage{pdflscape}
\usepackage{rotating}
\usepackage{caption}

%make landscape mode be sensitive to even and odd pages
%start
\def\myrotate{\ifodd\c@page\else-\fi 90}
\makeatletter
\global\let\orig@begin@landscape=\landscape%
\global\let\orig@end@landscape=\endlandscape%
\gdef\@true{1}
\gdef\@false{0}
\gdef\landscape{%
    \global\let\within@landscape=\@true%
    \orig@begin@landscape%
}%
\gdef\endlandscape{%
    \orig@end@landscape%
    \global\let\within@landscape=\@false%
}%
\@ifpackageloaded{pdflscape}{%
    \gdef\pdf@landscape@rotate{\PLS@Rotate}%
}{
    \gdef\pdf@landscape@rotate#1{}%
}
\let\latex@outputpage\@outputpage
\def\@outputpage{
    \ifx\within@landscape\@true%
        \if@twoside%
            \ifodd\c@page%
                \gdef\LS@rot{\setbox\@outputbox\vbox{%
                    \pdf@landscape@rotate{-90}%
                    \hbox{\rotatebox{90}{\hbox{\rotatebox{180}{\box\@outputbox}}}}}%
                }%
            \else%
                \gdef\LS@rot{\setbox\@outputbox\vbox{%
                    \pdf@landscape@rotate{+90}%
                    \hbox{\rotatebox{90}{\hbox{\rotatebox{0}{\box\@outputbox}}}}}%
                }%
            \fi%
        \else%
            \gdef\LS@rot{\setbox\@outputbox\vbox{%
                \pdf@landscape@rotate{+90}%
                \hbox{\rotatebox{90}{\hbox{\rotatebox{0}{\box\@outputbox}}}}}%
            }%
        \fi%
    \fi%
    \latex@outputpage%
}
\makeatother
%end

%graphics
\usepackage{graphicx}
\usepackage[dvipsnames, table]{xcolor}
%If you upload images from PC, you need to insert code for the path here (different for Windows and Unix OS)

%References
%\usepackage{xpatch}
%\usepackage[backend=biber, style=numeric, citestyle=numeric, sorting=none]{biblatex}
%\addbibresource{main.bib}

%Other
\usepackage{ifthen}
\usepackage{xspace}
\usepackage{enumitem}
\usepackage{amssymb}
\usepackage[pdftex, colorlinks]{hyperref}
%prevents internal hyperlinks default color (red), keeps links active but black
\hypersetup{%
  colorlinks = true,
  linkcolor  = black
}
\newcommand{\comment}[1]{{\color{Red}$\blacktriangleright$ Comment: #1 $\blacktriangleleft$}}

% Some utilities\ldots
\usepackage{soul}
\usepackage{tikz}

\usetikzlibrary{calc}
\usetikzlibrary{decorations.pathmorphing}


\makeatletter

\newcommand{\defhighlighter}[3][]{%
  \tikzset{every highlighter/.style={color=#2, fill opacity=#3, #1}}%
}

\defhighlighter{yellow}{.5}

\newcommand{\highlight@DoHighlight}{
  \fill [ decoration = {random steps, amplitude=1pt, segment length=15pt}
        , outer sep = -15pt, inner sep = 0pt, decorate
       , every highlighter, this highlighter ]
        ($(begin highlight)+(0,8pt)$) rectangle ($(end highlight)+(0,-3pt)$) ;
}

\newcommand{\highlight@BeginHighlight}{
  \coordinate (begin highlight) at (0,0) ;
}

\newcommand{\highlight@EndHighlight}{
  \coordinate (end highlight) at (0,0) ;
}

\newdimen\highlight@previous
\newdimen\highlight@current

\DeclareRobustCommand*\highlight[1][]{%
  \tikzset{this highlighter/.style={#1}}%
  \SOUL@setup
  %
  \def\SOUL@preamble{%
    \begin{tikzpicture}[overlay, remember picture]
      \highlight@BeginHighlight
      \highlight@EndHighlight
    \end{tikzpicture}%
  }%
  %
  \def\SOUL@postamble{%
    \begin{tikzpicture}[overlay, remember picture]
      \highlight@EndHighlight
      \highlight@DoHighlight
    \end{tikzpicture}%
  }%
  %
  \def\SOUL@everyhyphen{%
    \discretionary{%
      \SOUL@setkern\SOUL@hyphkern
      \SOUL@sethyphenchar
      \tikz[overlay, remember picture] \highlight@EndHighlight ;%
    }{%
    }{%
      \SOUL@setkern\SOUL@charkern
    }%
  }%
  %
  \def\SOUL@everyexhyphen##1{%
    \SOUL@setkern\SOUL@hyphkern
    \hbox{##1}%
    \discretionary{%
      \tikz[overlay, remember picture] \highlight@EndHighlight ;%
    }{%
    }{%
      \SOUL@setkern\SOUL@charkern
    }%
  }%
  %
  \def\SOUL@everysyllable{%
    \begin{tikzpicture}[overlay, remember picture]
      \path let \p0 = (begin highlight), \p1 = (0,0) in \pgfextra
        \global\highlight@previous=\y0
        \global\highlight@current =\y1
      \endpgfextra (0,0) ;
      \ifdim\highlight@current < \highlight@previous
        \highlight@DoHighlight
        \highlight@BeginHighlight
      \fi
    \end{tikzpicture}%
    \the\SOUL@syllable
    \tikz[overlay, remember picture] \highlight@EndHighlight ;%
  }%
  \SOUL@
}

\makeatother

% Common abbrev. are set as commands to ensure proper spacing after the dot
\RequirePackage{xspace}
\newcommand{\ie}{i.e.\@\xspace}
\newcommand{\aka}{a.k.a.\@\xspace}
\newcommand{\Ie}{I.e.\@\xspace}
\newcommand{\cf}{cf.\@\xspace}
\newcommand{\Cf}{Cf.\@\xspace}
\newcommand{\eg}{e.g.\@\xspace}
\newcommand{\Eg}{E.g.\@\xspace}
\newcommand{\etal}{et al.\@\xspace}
\newcommand{\etc}{etc.\@\xspace}
\newcommand{\wrt}{w.r.t.\@\xspace}
\newcommand{\Wrt}{W.r.t.\@\xspace}

\date{}

%to include alloy source code
\usepackage{listings}
\lstdefinelanguage{alloy}{
  keywords={%
      assert, pred, all, no, lone, one, some, check, run,
      but, let, implies, not, iff, in, and, or, set, sig, Int, int,
      if, then, else, exactly, disj, fact, fun, module, abstract,
      extends, open, none, univ, iden, seq,
      for, as, sum,
  },
  literate=%
    *{:}{{{\color[HTML]{2835C0}{$\colon$}}}}1
    {>}{{{\color[HTML]{2835C0}{>}}}}1
    {<}{{{\color[HTML]{2835C0}{<}}}}1
    {|}{{{\color[HTML]{2835C0}{|}}}}1
    {==}{{{\color[HTML]{2835C0}{$=$}}}}1
    {=}{{{\color[HTML]{2835C0}{$=$}}}}1
    {!=}{{{\color[HTML]{2835C0}{$\neq$}}}}1
    {&&}{{{\color[HTML]{2835C0}{$\land$}}}}1
    {||}{{{\color[HTML]{2835C0}{$\lor$}}}}1
    {<=}{{{\color[HTML]{2835C0}{$\le$}}}}1
    {>=}{{{\color[HTML]{2835C0}{$\ge$}}}}1
    {!in}{{{\color[HTML]{2835C0}{$\not\in$}}}}1
    {\\in}{{{\color[HTML]{2835C0}{$\in$}}}}1
    {=>}{{{\color[HTML]{2835C0}{$\implies$}}}}2
    % the following isn't actually Alloy, but it gives the option to produce nicer latex
    {|=>}{{{\color[HTML]{2835C0}{$\Rightarrow$}}}}2
    {<=set}{{{\color[HTML]{2835C0}{$\subseteq$}}}}1
    {+set}{{{\color[HTML]{2835C0}{$\cup$}}}}1
    {*set}{{{\color[HTML]{2835C0}{$\cap$}}}}1
    {==>}{{{{\color[HTML]{2835C0}$\Longrightarrow$}}}}3
    {<==>}{$\Longleftrightarrow$}4
    {...}{$\ldots$}1
    {\\hl}{$\hline$}1
    {\\alpha}{$\alpha$}1
    {\\beta}{$\beta$}1
    {\\gamma}{$\gamma$}1
    {\\delta}{$\delta$}1
    {\\epsilon}{$\epsilon$}1
    {\\zeta}{$\zeta$}1
    {\\eta}{$\eta$}1
    {\\theta}{$\theta$}1
    {\\iota}{$\iota$}1
    {\\kappa}{$\kappa$}1
    {\\lambda}{$\lambda$}1
    {\\mu}{$\mu$}1
    {\\nu}{$\nu$}1
    {\\xi}{$\xi$}1
    {\\pi}{$\pi$}1
    {\\rho}{$\rho$}1
    {\\sigma}{$\sigma$}1
    {\\tau}{$\tau$}1
    {\\upsilon}{$\upsilon$}1
    {\\phi}{$\phi$}1
    {\\chi}{$\chi$}1
    {\\psi}{$\psi$}1
    {\\omega}{$\omega$}1
    {\\Gamma}{$\Gamma$}1
    {\\Delta}{$\Delta$}1
    {\\Theta}{$\Theta$}1
    {\\Lambda}{$\Lambda$}1
    {\\Xi}{$\Xi$}1
    {\\Pi}{$\Pi$}1
    {\\Sigma}{$\Sigma$}1
    {\\Upsilon}{$\Upsilon$}1
    {\\Phi}{$\Phi$}1
    {\\Psi}{$\Psi$}1
    {\\Omega}{$\Omega$}1
    {\\EOF}{\;}1
    ,
  sensitive=true,  % case sensitive
  morecomment=[l]//,%
  morecomment=[l]{--},%
  morecomment=[s]{/*}{*/},%
  morestring=[b]",
  numbers=none,
  firstnumber=1,
  numberstyle=\tiny,
  stepnumber=2,
  basicstyle=\footnotesize\ttfamily,
  commentstyle=\color[HTML]{00A108}\itshape,
  keywordstyle=\color[HTML]{2835C0}\bfseries,
  ndkeywordstyle=\bfseries,
}

% inline
\def\A{%
    \lstinline[language=alloy,basicstyle=\ttfamily,columns=fixed]}

% paragraph
\lstnewenvironment{alloy}[1][]{%
  \lstset{language=alloy,
    floatplacement={tbp},captionpos=b,
    xleftmargin=8pt,xrightmargin=8pt,basicstyle=\ttfamily,#1}}{}

% paragraph from file
\newcommand{\alloyfile}[1]{
  \lstinputlisting[language=alloy,%
    frame=lines,xleftmargin=8pt,xrightmargin=8pt,basicstyle=\footnotesize\ttfamily,columns=fixed]{#1}
}

\lstset{
    %Add this if you want to display border around your code  
    frame=single,
    breaklines=true,
    postbreak=\raisebox{0ex}[0ex][0ex]{\ensuremath{\color{red}\hookrightarrow\space}}
}