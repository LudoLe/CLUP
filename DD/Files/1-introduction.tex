\section{Introduction}
\label{sect:introduction}

\subsection{Purpose}
\label{subsect:purpose}
This document is mainly addressed to the developement software team. 
Its purpose, indeed, is to provide the reader of an overview of the architecture of the software product.

\subsection{Scope}
\label{subsect:scope}

Clup is a digital application which will provide the following services to the clients:
\begin{itemize}
    \item Browse of registered supermarkets
    \item Join a virtual queue of registered supermarkets 
    \item Book up a shopping session in registered supermarkets 
    \item Browse of previous and future shopping sessions 
\end{itemize}
Clup is a digital application which will provide the following services to the shop-managers:
\begin{itemize}
    \item Registration of their shop on the system
    \item Management of the registered shops
\end{itemize}


\subsection{Definitions, acronyms and abbreviations}
\label{subsect:definitionsacronymsabbreviations}

\textit{acronyms} and \textit{abbreviations} used throughout the document.

\begin{table}[h!]
    \centering
    \begin{tabular}{@{}P{0.25\textwidth}P{0.6\textwidth}@{}}
        \multicolumn{2}{c}{\textbf{Definitions, Abbreviations, Acronyms}}\\
        \toprule
        \textbf{API} & Application Programming Interface\\
        \textbf{DD} & Design DOcument\\
        \textbf{JPA} & a grocery shop owner or administrator who hasn't already adhered to CLup\\
        \textbf{BCE} & Business Controller Entity\\
        \textbf{AJP} & Apache JServ Protocol\\
        \textbf{JPA} & Java Persistence API\\
        \textbf{RASD} & Requirements Analysis and Specifications Document\\
        \textbf{UX} & User Experience\\
        \textbf{JS} & JavaScript\\
        \textbf{JDBC} & Java DataBase connection\\
        \textbf{FIFO} & First in, first out\\
        \textbf{DMZ} & Demilitarized zone\\
        \textbf{IaaS} & Infrastracture as a Service\\
        \textbf{ELB} & Elastic Load Balancer\\

    \end{tabular}
\caption{Definitions}
\label{table:definitionsabbreviationsacronyms}
\end{table}
\textbf{Definition}:
A disambiguation is here needed, with the term \textbf{"queue"}, here and in every other part of this document, we indicate a series of tickets in temporal order, either queue tickets or visit tickets. 

However, from time to time, the terms \textit{"queue", "dequeue", "enqueuement", "dequement"} will be referred only to users who decide to line up in the \textbf{virtual queue}, intended as the one composed only of queue tickets. 

Even though it may seems confusing now, we have retained that the context will be enough of a disclaimer for the user to have clearly in mind which kind of queue we are talking about.

\subsection{Referenced documents}
\label{subsect:referenceddocuments}
\begin{itemize}
    \item Project goal, schedule and rules (Assignements AA. 2020/21)
    \item Requirements Analysis and Specifications Document v.2
    \item UML component and deployement diagram\\
    uml-diagrams.org
    \item Java client library for Google Maps API Web Services documentation\\
    github.com/googlemaps/google-maps-JavaGoogle Maps Distance Matrix API documentation\\
\end{itemize}

\subsection{Document structure}
\label{subsect:document structure}
\begin{itemize}
    \item \textbf{1 Introduction} 
    \item \textbf{2 Architectural design}
    \item \textbf{3 Algorithm design}
    \item \textbf{4 User Interface design}
    \item \textbf{5 Requirements traceability}
\end{itemize}


