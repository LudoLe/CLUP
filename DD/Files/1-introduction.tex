\section{Introduction}
\label{sect:introduction}

\subsection{Purpose}
\label{subsect:purpose}
This document is mainly addressed to the developement software team. 
Its purpose, indeed, is to provide the reader of an overview of the architecture and design choices of the software product.

\subsection{Scope}
\label{subsect:scope}

Clup is a digital application which will provide the following services to the clients:
\begin{itemize}
    \item Browse of registered supermarkets
    \item Join a virtual queue of registered supermarkets 
    \item Book up a shopping session in registered supermarkets 
    \item Browse of previous and future shopping sessions 
\end{itemize}
Clup is a digital application which will provide the following services to the shop-managers:
\begin{itemize}
    \item Registration of their shop on the system
    \item Management of the registered shops
\end{itemize}


\subsection{Definitions, acronyms and abbreviations}
\label{subsect:definitionsacronymsabbreviations}

With the term \textbf{queue}, a disambiguation is needed: here and in every other part of this document, we indicate a series of tickets in temporal order, either queue tickets or visit tickets. 

However, from time to time, the terms \textit{queue, dequeue, enqueuement, dequement} will be referred only to users who decide to line up in the \textbf{virtual queue}, intended as the one composed only of queue tickets. 

Even though it may seem confusing now, we have retained that the context will be enough of a disclaimer for the user to have clearly in mind which kind of queue we are talking about.

\noindent
The following table [\ref{table:acronyms}] shows all the acronyms used in the document.

\begin{table}[ht!]
    \centering
    \begin{tabular}{@{}P{0.25\textwidth}P{0.6\textwidth}@{}}
        \multicolumn{2}{c}{\textbf{Acronyms}}\\
        \toprule
        \textbf{API} & Application Programming Interface\\
        \textbf{DD} & Design DOcument\\
        \textbf{JPA} & a grocery shop owner or administrator who hasn't already adhered to CLup\\
        \textbf{BCE} & Business Controller Entity\\
        \textbf{AJP} & Apache JServ Protocol\\
        \textbf{JPA} & Java Persistence API\\
        \textbf{RASD} & Requirements Analysis and Specifications Document\\
        \textbf{UX} & User Experience\\
        \textbf{JS} & JavaScript\\
        \textbf{JDBC} & Java DataBase connection\\
        \textbf{FIFO} & First in, first out\\
        \textbf{DMZ} & Demilitarized zone\\
        \textbf{IaaS} & Infrastracture as a Service\\
        \textbf{ELB} & Elastic Load Balancer\\
    \end{tabular}
\caption{Definitions}
\label{table:acronyms}
\end{table}

\FloatBarrier

%TODO: (lo scrivo qua anche se non è il posto adatto...) dobbiamo parlare del fatto che i totem possono avere problemi di ritardo e che quindi serve uno schermetto.

