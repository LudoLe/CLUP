\section{Queue algorithm}
\label{sect:algorithm}

\subsection{General overview}
\label{subsect:introduction}

The \textit{management of shop entry turns} is a fundamental objective of the system in order to have it functioning properly, because most of the actions a user performs is related in some way to the stores queues.

Before we start discussing the queue algorithm, we need to define what we mean by the term "\textit{queue}" in this section: a \textbf{queue} is a data structure representing a time line that contains all the necessary information regarding the entry times of all those customers who have queued or booked a shopping session.

Queues are objects that change over time based on unpredictable events and uncertain data. Our aim is to create queues and estimates that are as reliable as possible, to have solution for edge cases, and to be efficient not wasting the store and client's time.

\subsubsection{Input}
\label{subsubsect:input}

To create a queue the only inputs we need are listed in the following table [\ref{table:input}].

\begin{table}[h!]
    \centering
    \begin{tabular}{@{}P{0.2\textwidth}P{0.78\textwidth}@{}}
        \toprule
        \textbf{Input name}& \textbf{Description}\\
        \midrule
        \textbf{Store capacity}           & Parameter that indicates the maximum amount of people that the store accepts at the same time inside it in a instant of time.\\
        \textbf{People inside the shop}   & A list of all the tickets that are currently in the store.\\
        \textbf{Queued tickets}           & A list of all the tickets of those that have queued in the shop, and haven't entered yet.\\
        \textbf{Visit tickets}            & A list of all the tickets of those that have booked a visit in the shop, and haven't entered yet.\\
        \textbf{Search radius for visits} & Parameter that indicates the amount of time within which to hold a free entrance to the shop in view of a person who has booked a visit.\\
        \bottomrule
    \end{tabular}
\caption{Queue algorithm input}
\label{table:input}
\end{table}

\FloatBarrier

For the algorithm to work properly, we must make sure that the data provided are coherent and correct, the algorithm does not perform any type of checks or validation.

\subsubsection{Expected behavior}
\label{subsubsect:expectedbehavior}

To understand the algorithm it is necessary to explain what are the most important rules on which the queues are based:

\begin{itemize}[topsep=0pt]
    \item When possible, the algorithm will use a \textit{FIFO policy}.
    \item Each store has a \textit{maximum capacity} of people that it can accommodate inside at the same time.
    \item \textit{For each person who leaves the store, one can enter it}. This is the most important principle for achieving a constant and optimized flow of people within a store. To fully understand the logic behind this algorithm, it is important to highlight the fact that there must always be a \textbf{one-to-one correspondence} between a ticket leaving the store and a ticket entering the store.\newline
    More formally we can say that there are two options available, if at time $t$ a ticket leaves the store, a \textit{one-to-one correspondence} will be created with:
    \begin{enumerate}[topsep=0pt]
        \item a visit ticket, if there is one whose entry is scheduled within the time $t + range$ (with “$range$” defined in the input table [\ref{table:input}]);
        \item a queue ticket, that will be scheduled at time $t + 1$;
    \end{enumerate}
    \item Customers who book a visit have \textit{priority} over customers who queue in the shop. However, giving maximum priority to visit tickets can result in the creation of inefficient queues, so the key to managing this aspect is to find the right balance between priority and efficiency, with a correct value for the input parameter \textit{Search radius for visits}.
    \item The algorithm need to make sure that every person with a queue ticket is not scheduled ahead of the \textit{time} it takes \textit{to react the store}.
\end{itemize}

With these rules the algorithm will cover the main functionalities, but more features can easily be added, an example can be the possibility to set a maximum amount of people that can enter a slot in a single time slot.

%TODO: what about stores areas

\subsubsection{Output}
\label{subsubsect:Output}

\begin{table}[h!]
    \centering
    \begin{tabular}{@{}P{0.13\textwidth}P{0.85\textwidth}@{}}
        \toprule
        \textbf{Output name}& \textbf{Description}\\
        \midrule
        \textbf{Queue}                & A timeline divided into discrete time slots, each containing information regarding the tickets that are expected to enter the store in that time slot, and the tickets that are expected to leave the store in that time slot.\\
        \textbf{Estimates}            & General queue statistics, such as the estimated queue length, will be provided so that they are easy to access.\\
        \bottomrule
    \end{tabular}
\caption{Queue algorithm output}
\label{table:output}
\end{table}

\subsection{Integration of the algorithm with the system}
\label{subsubsect:integrationofthealgorithmintothesystem}

In our system it is quite common for a process to use the data related to a store queue, for this reason the result of the algorithm will be saved in the database in order to make it available to any process that needs it. Every time the queue must be modified, the algorithm will build a new queue from scratch and override the old one in the database. This decision is in line with the application workflow, since most events will only need to read the queue data, and not to modify them, and therefore we expect to not overload the servers.

To have a functioning system, however, the queue must be always updated. To accomplish this, we have designed a life cycle for the algorithm based on the reactions to all those events that can modify the queue:

\begin{enumerate}[topsep=0pt]
    \item An event that could change the queue occurs, and the relative data in the database are updated.
    \item The currently existing queue in the database is discarded.
    \item The algorithm runs with the new updated data and produce a new queue.
    \item The new queue is saved in the database, replacing the old one.
\end{enumerate}

\noindent
The following is a list of all the events that may cause a change in the queue:

\begin{itemize}[topsep=0pt]
    \item someone lines up;
    \item someone books a visit;
    \item someone scans a QR-code to enter or exit a store;
    \item someone cancels a visit or an enqueuement;
\end{itemize}

It is important to point out that there is a particular event, which, however, was not mentioned in the list above: \textit{someone doesn't show up}. Due to the way the algorithm life cycle has been structured, this event resolves itself automatically. In fact, when someone does not show up at the expected time, the information can be ignored: when the next event happens, a new queue will be built and the person who did not show up previously will not be included among the inputs of the algorithm, and consequently will no longer be part of the queue.

\subsection{The algorithm}
\label{subsubsect:implementation}

In this section we want to show how the algorithm works describing it step-by-step in natural language and to show a possible implementation in Java code.

\subsubsection{Natural language}
\label{subsubsect:naturallanguage}

The goal of this section is to explain the logical steps of the queue algorithm while ignoring the technical details, so that we can focus on the overall structure.\newline Following there are explained all the steps needed to obtain a queue that meets the requirements of the system:

\begin{enumerate}[topsep=0pt]
    \item Create an empty time line, made up of discrete time slots.
    \item If the shop isn't full, the amount of people that can enter the shop in in the initial time is set to the number of available entrances.
    \item Loops through the list of all the people inside the store, and for each one of them set the estimated exit time in the timeline.
    \item For each visit ticket, set its entrance in the time slot in which the ticket is scheduled for, and set its exit in the time slot in which its exit is estimated.
    \item Now we need to find a solution to create a \textit{one-to-one correspondence} for entering tickets with exiting tickets.\newline
          Loops through all slots of the timeline and for each person who exits:
          \begin{itemize}
              \item search within the established "range" (see algorithm input table [\ref{table:input}]) if there is a scheduled visit ticket whose one-to-one correspondence hasn't already been solved.\newline
              If a visit ticket is found, then we consider the ticket solved, and we set it in one-to-one correspondence with the person leaving the shop.
              \item Else, if it is possible, we mark the person leaving the store in one-to-one correspondence with the first queue ticket that manages to show up in time, using the FIFO rule to decide which to solve.
          \end{itemize}
\end{enumerate}

An important aspect of this algorithm is that more features can be easily added without changing the main logic steps.

\subsubsection{Java code}
\label{subsubsect:javacode}

This section shows a possible implementation of the algorithm in Java.

The following implementation is not to be considered a final nor complete version, its only meant to be a guide for future developers to better understand the logic. Furthermore, the code is structured to be easy to read, excessively commented, inefficient, and does not respect good coding conventions.

The following code is divided into three Java classes: \textit{Queue, Ticket, TimeSlot}.

The TimeSlot and Ticket classes do not contain any algorithm logic, they are purely used to contain data in an orderly and sensible way. The \textbf{TimeSlot} class is useful for building the timeline, while the \textbf{Ticket} class is used to represent people queued, or who have booked a visit, or who are already inside the store.

The output of the algorithm is contained within the \textbf{Queue} class, which also contains all the logic within the Queue.buildQueue () method.


\subsubsection*{TimeSlot.java}
\label{subsubsect:timeslotjava}

\javaFile{algorithmcode/TimeSlot.java}

\subsubsection*{Ticket.java}
\label{subsubsect:ticketjava}

For ease of exposure and to reduce the amount of code within this document, this class is used to represent both queue tickets and visit tickets, although they should have different properties and methods. A more sensible hierarchy is expected in the actual application.\newline
\newline

\javaFile{algorithmcode/Ticket.java}

\subsubsection*{Queue.java}
\label{subsubsect:queuejava}

This class takes care of the logic of the algorithm thanks to the .buildQueue () method, of storing the output, and of providing methods that allow easy access to the most important estimates.

\javaFile{algorithmcode/Queue.java}